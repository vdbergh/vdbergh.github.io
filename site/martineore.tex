\documentclass{amsart}
%\usepackage{showkeys}
\usepackage{amscd,verbatim}
\numberwithin{equation}{section}

%  amstexincl.tex
%

%\input{scr}
%
%  scr.tex
%
\let\cal\mathcal
\def\Ascr{{\cal A}}
\def\Bscr{{\cal B}}
\def\Cscr{{\cal C}}
\def\Dscr{{\cal D}}
\def\Escr{{\cal E}}
\def\Fscr{{\cal F}}
\def\Gscr{{\cal G}}
\def\Hscr{{\cal H}}
\def\Iscr{{\cal I}}
\def\Jscr{{\cal J}}
\def\Kscr{{\cal K}}
\def\Lscr{{\cal L}}
\def\Mscr{{\cal M}}
\def\Nscr{{\cal N}}
\def\Oscr{{\cal O}}
\def\Pscr{{\cal P}}
\def\Qscr{{\cal Q}}
\def\Rscr{{\cal R}}
\def\Sscr{{\cal S}}
\def\Tscr{{\cal T}}
\def\Uscr{{\cal U}}
\def\Vscr{{\cal V}}
\def\Wscr{{\cal W}}
\def\Xscr{{\cal X}}
\def\Yscr{{\cal Y}}
\def\Zscr{{\cal Z}}
%\input {blb}
%
%  blb.tex
%
\let\blb\mathbb
\def\CC{{\blb C}}
\def\XX{{\blb X}}
\def\FF{{\blb F}}
\def\QQ{{\blb Q}}
\def\GG{{\blb G}}
\def \PP{{\blb P}}
\def \AA{{\blb A}}
\def \ZZ{{\blb Z}}
\def \TT{{\blb T}}
\def \NN{{\blb N}}
\def \RR{{\blb R}}
\def \HH{{\blb H}}

\def\qch{\text{-qch}}
\def\wqch{\text{-wqch}}
\def\pwqch{\text{-(w)qch}}
\def\Ann{\operatorname{Ann}}
\def\Nil{\operatorname{Nil}}
\def\Nil{\operatorname{Nil}}
\newcommand{\proj}{\operatorname{proj}}
\def\id{\text{id}}
\def\Id{\operatorname{id}}
\def\pr{\mathop{\text{pr}}\nolimits}
\let\st\ast
\let\at\ast
\def\Der{\operatorname{Der}}
\def\Bimod{\operatorname{Bimod}}
\def\Ab{\mathbb{Ab}}
\def\Tau{\mathcal{T}}
\def\Res{\operatorname{Res}}
\def\ctimes{\mathbin{\hat{\otimes}}}
\def\Lotimes{\overset{L}{\otimes}}
\def\quot{/\!\!/}
\def\ctensor{\mathbin{\hat{\otimes}}}
\def\Mod{\operatorname{Mod}}
\def\mod{\operatorname{mod}}
\def\Gr{\operatorname{Gr}}
\def\GR{\operatorname{GR}}
\def\QGr{\operatorname{QGr}}
\def\qgr{\operatorname{qgr}}
\def\Hol{\text{-Hol}}
\def\gr{\operatorname{gr}}
\def\Lie{\mathop{\text{Lie}}}
\def\Ch{\mathop{\mathrm{Ch}}}
\def\CH{\mathop{\mathrm{CH}}}
\def\length{\mathop{\text{length}}}
\def\ICh{\mathop{\text{ICh}}}
\def\Supp{\mathop{\text{\upshape Supp}}}
\def\ch{\mathop{\text{Ch}}}
\def\QSch{\mathop{\text{Qsch}}}
\def\Qch{\mathop{\text{Qch}}}
\def\coh{\mathop{\text{\upshape{coh}}}}
\def\charact{\operatorname{char}}
\def\adm{\operatorname{adm}}
\def\spec{\operatorname {Spec}}
\def\coop{{\operatorname {coop}}}
\def\rad{\operatorname {rad}}
\def\PC{\operatorname {PC}}
\def\dis{\operatorname {dis}}
\def\pc{\operatorname {pc}}
\def\qpc{\operatorname {qpc}}
\def\gr{\operatorname {gr}}
\def\Spec{\operatorname {Spec}}
\def\Rep{\operatorname {Rep}}
\def\GL{\operatorname {GL}}
\def\PGL{\operatorname {PGL}}
\def\depth{\operatorname {depth}}
\def\diag{\operatorname {diag}}
\def\Ext{\operatorname {Ext}}
\def\Hom{\operatorname {Hom}}
\def\End{\operatorname {End}}
\def\RHom{\operatorname {RHom}}
\def\uRHom{\operatorname {R\mathcal{H}\mathit{om}}}
\def\Sl{\operatorname {Sl}}
\def\cd{\operatorname {cd}}
\def\codim{\operatorname {codim}}
\def\supp{\operatorname {supp}}
\def\relint{\operatorname {relint}}
\def\im{\operatorname {im}}
\def\tr{\operatorname {Tr}}
\def\Tr{\operatorname {Tr}}
\def\coker{\operatorname {coker}}
\def\ker{\operatorname {ker}}
\def\Ker{\operatorname {ker}}
\def\Tor{\operatorname {Tor}}
\def\End{\operatorname {End}}
\def\Gal{\operatorname {Gal}}
\def\for{\operatorname {For}}
\def\tot{\operatorname {Tot}}
\def\add{\operatorname {add}}
\def\rk{\operatorname {rk}}
\def\For{\operatorname {For}}
\def\Tot{\operatorname {Tot}}
\def\ord{\operatorname {ord}}
\def\Pic{\operatorname {Pic}}
\def\gkdim{\operatorname {GK dim}}
\def\Filt{\operatorname {Filt}}
\def\gldim{\operatorname {gl\,dim}}
\def\G{\mathop{\underline{\underline{{\Gamma}}}}\nolimits}
\def\r{\rightarrow}
\def\l{\leftarrow}
\def\u{\uparrow}
\def\d{\downarrow}
\def\exist{\exists}
\DeclareMathOperator{\Proj}{Proj}
\DeclareMathOperator{\Alg}{Alg}
\DeclareMathOperator{\GrAlg}{GrAlg}
\DeclareMathOperator{\Rel}{Rel}
\DeclareMathOperator{\Tors}{Tors}
\DeclareMathOperator{\tors}{tors}
\DeclareMathOperator{\HTor}{\mathcal{T}\mathit{or}}
\DeclareMathOperator{\HHom}{\mathcal{H}\mathit{om}}
\DeclareMathOperator{\HEnd}{\mathcal{E}\mathit{nd}}
\DeclareMathOperator{\Nrd}{Nrd}
\DeclareMathOperator{\Aut}{Aut}
\DeclareMathOperator{\ram}{ram}
\DeclareMathOperator{\Br}{Br}
\DeclareMathOperator{\Div}{Div}
\DeclareMathOperator{\Prin}{Prin}
\DeclareMathOperator{\cor}{cor}
\DeclareMathOperator{\res}{res}


\let\dirlim\injlim


%
%  sectionlemma.tex
%
% theoremstyle plain

\newtheorem{lemma}{Lemma}[section]
\newtheorem{proposition}[lemma]{Proposition}
\newtheorem{theorem}[lemma]{Theorem}
\newtheorem{corollary}[lemma]{Corollary}
\newtheorem{condition}[lemma]{Condition}
\newtheorem{exercise}[lemma]{Exercise}
\newtheorem{conclusion}[lemma]{Conclusion}
\newtheorem{lemmas}{Lemma}[subsection]
\newtheorem{propositions}[lemmas]{Proposition}
\newtheorem{theorems}[lemmas]{Theorem}
\newtheorem{corollarys}[lemmas]{Corollary}
\newtheorem{conditions}[lemmas]{Condition}
\newtheorem{conclusions}[lemmas]{Conclusion}



\newtheorem{sublemma}{Sublemma}\def\thesublemma{}
\newtheorem{subcorollary}{Corollary}
\theoremstyle{definition}

\newtheorem{example}[lemma]{Example}
\newtheorem{definition}[lemma]{Definition}
\newtheorem{conjecture}[lemma]{Conjecture}
\newtheorem{question}[lemma]{Question}
\newtheorem{examples}[lemmas]{Example}
\newtheorem{definitions}[lemmas]{Definition}
\newtheorem{conjectures}[lemmas]{Conjecture}
\newtheorem{questions}[lemmas]{Question}
{
\newtheorem{claim}{Claim}
\newtheorem{step}{Step}
\newtheorem{case}{Case}
\newtheorem*{answer}{Answer}
\newtheorem{hypothesis}{Hypothesis}
}

\theoremstyle{remark}

\newtheorem{remark}[lemma]{Remark}
\newtheorem{remarks}[lemmas]{Remark}
\newtheorem{notation}{Notation}\def\thenotation{}

\newdimen\uboxsep \uboxsep=1ex
\def\uboxn#1{\vtop to 0pt{\hrule height 0pt depth 0pt\vskip\uboxsep
\hbox to 0pt{\hss #1\hss}\vss}}

\def\uboxs#1{\vbox to 0pt{\vss\hbox to 0pt{\hss #1\hss}
\vskip\uboxsep\hrule height 0pt depth 0pt}}



\def\debug{\tracingall}
\def\enddebug{\showlists}





\author{Michel Van den Bergh and Martine Van Gastel}

\address{Departement WNI\\ Limburgs Universitair
Centrum\\ Universitaire Campus\\ Building D\\ 3590
Diepenbeek\\ Belgium}

\email[M. Van Gastel]{martine.vangastel@luc.ac.be}
\email[M. Van den Bergh]{michel.vandenbergh@luc.ac.be}

\title[Non-commutative regular local rings]{On the structure of non-commutative regular local rings of
  dimension two}

\keywords{Ore extensions}
\subjclass{Primary 16L30}


\begin{document}

\begin{abstract}
In this paper we conjecture that the center of a non-commutative
complete regular local ring of global dimension two is a formal power
series ring in two variables. We prove this conjecture in the special
case of Ore extensions. 
\end{abstract}

\maketitle

\section{Introduction} \label{ref-1-0}

Below $k$ is a field. 
In this paper we will be concerned with  rings of the form 
$C=k\langle \langle x,y
\rangle \rangle/(r)$ where $r$ only has term of total degree $\ge
2$ and where the quadratic part of $r$ is  non-degenerate. Such rings have global dimension two \cite{VdBVG} and it may be
argued that they are the non-commutative analogues of two-dimensional
regular local rings.

In this paper we propose the following conjecture:
\begin{conjecture}
\label{ref-1.1-1}
Let $C$ be as above. Then the  center of $C$ is either trivial,
or else it is a formal power series ring in two variables. If the
quadratic part of $r$ is of the form $yx-xy$   and the
characteristic  $p$ of $k$ is $>0$ then $Z(C)$ is generated by
elements  of the form $x^{p^n}+\cdots$ and $y^{p^n}+\cdots$ for some $n>0$.
\end{conjecture}
\begin{comment}
\begin{conjecture}
    Let $C$ be a non-commutative surface
    \[k\langle \langle x,y \rangle \rangle /(yx\;-\;qxy\;-\;\psi(x,y))\]
    where $q \in k^{*}$ and where the total degree of $\psi(x,y) \mbox{
    is } \geq 3$.
    \\Then $C$ is finite over its center $Z(C)$ and
    $Z(C)\:=\:k[[\alpha,\beta]]$, where $\alpha \: = \: x^{p^{m}}\: 
    +\: \varphi(x)$ and $\beta\:=\:y^{p^{m}}\:+\:\theta(x,y)$, for some $m \in \NN$ 
    and $\varphi(x),\;\theta(x,y)$ have only terms of total degree $>p^{m}$.
\end{conjecture}
\end{comment}
In this paper we will provide some evidence for this conjecture by
proving it  in the case that $C$ is given by an Ore extension
$C=B[[y;\sigma,\delta]]$ where $B$ is $k[[x]]$, $\sigma$ is a
$k$-linear automorphism of $B$ and $\delta$ is a $k$-linear
$\sigma$-derivation of $B$. Thus $\delta$ satisfies
$\delta(ab)\:=\:\sigma(a)\delta(b)\:+\:\delta(a)b$ and $C$ is obtained
from $B$ by adjoining the variable $y$ subject to the commutation rule
\begin{equation} \label{ref-1.1-2}
    y\,b\;=\;\sigma(b)y\;+\;\delta(b)
\end{equation}
In other words $C=k\langle\langle x,y\rangle\rangle/(r)$ where $r$ is given
by $yx-\sigma(x)y-\delta(x)$. Thus for $r$ to have only terms of
degree $\ge 2$ it is necessary that $\delta(x)$ contains only terms
of degree $\ge 2$. We assume this throughout.



We will prove the following theorem:

\begin{theorem} \label{ref-1.2-3}
If $C$ is an Ore extension as above then Conjecture
\ref{ref-1.1-1} holds.
\end{theorem}
\begin{comment}
\begin{enumerate} 
\item
$C$ has non-trivial center if and only if one of the following
conditions holds.
\begin{enumerate}
\item $\sigma$ and $\delta$ are non-trivial.
\item $\sigma$ is trivial, $\delta$ is not trivial  and $p>0$.
\item The order of $\sigma$ is finite and $\ge 2$.
\end{enumerate}


    Assume that $\sigma$ has finite order and that
    $\delta(x)\,=\,\lambda x^{n}\;+$ higher terms in $x$,where $n \geq 3$.
    \\ In that case $C$ will be finite over its center $Z(C)$ and
    $Z(C)\:=\:k[[z,w]]$, where $z\:=\:x^{p^{m}}\:+\:\varphi(x)$ and
    $w\:=\:y^{p^{m}}\:+\:\theta(x,y)$, for some $m \in \NN$ and
    $\varphi(x),\;\theta(x,y)$ have only terms of total degree
    $>p^{m}$.
\end{theorem}
\end{comment}





Our treatment of the case where $\sigma$ is trivial relied originally on the
following combinatorial result by  G.~Baron and A.~Schinzel in \cite{BS1}.
\begin{proposition} \label{ref-1.3-4}
    For any prime $p$ and any residues $x_{i}$ mod $p$, we have:
    \[\sum_{\sigma \in S_{p-1}}
    x_{\sigma(1)}(x_{\sigma(1)}\:+\:x_{\sigma(2)}) \ldots
    (x_{\sigma(1)}\:+ \ldots +\:x_{\sigma(p-1)}) \]
    \[ \equiv (x_{1}\:+ \cdots +\:x_{p-1})^{p-1} \hspace{1cm}
    (\operatorname{mod}\;p) \]
    where $S_{p-1}$ is the group of all permutations $\sigma$ of
    $\{1,\ldots,p-1\}$.
\end{proposition}
Afterwards we discovered a new approach which is independent of the above
result. It turns out that we can now even give a new proof of the
result by  G.~Baron and A.~Schinzel. This proof is produced in the
final section of this paper. Whereas the proof in \cite{BS1} is rather
technical, our proof is straightforward and relies on general
computations with derivations. 

\section{Outline}
In this section we outline our strategy for proving Theorem
\ref{ref-1.2-3}. First we dispense with some trivial cases. If $\sigma$
is trivial and $\delta=\Id$ then there is nothing to prove. 
In addition it is easy to prove that in the following cases the
center of $C$ is trivial.
\begin{enumerate}
\item $\sigma$ is trivial, $\delta$ is not trivial  and $p=0$.
\item The order of $\sigma$ is infinite.
\end{enumerate}
In subsequent sections we deal with the remaining cases.
 In Section~\ref{ref-3-5} we discuss the case where $\sigma$ is 
the identity and $p>0$.
In Section~\ref{ref-4-8} we focus on the case where $\delta$ 
is trivial and $\sigma$ is not trivial but has finite order. Finally in  Section~\ref{ref-5-11} 
 we deal with the case where both $\sigma$ and $\delta$ are
 non-trivial and $\sigma$ has finite order. In this last case our
 approach  is somewhat
 indirect  and we do not obtain nice expressions for the
 elements generating the center.


\section{The case where $\sigma$ is the identity and $p>0$}
\label{ref-3-5}

It follows  from (\ref{ref-1.1-2}) that in this case the commutation
relation between $y$ and $x$ is given by

\begin{equation} \label{ref-3.1-6}
    y\,x\;=\;x\,y\;+\;\delta(x)
\end{equation}

In this case we prove that $Z(C)$ equals $k[[z,w]]$, where
$z\,=\,x^{p}$ and \\ $w\,=\,y^{p}\,-\,c_{p}(x)\,y$, with
$\displaystyle{c_{p}(x)\:=\:\frac{\partial}{\partial x}
\left( \frac{\partial}{\partial x} \left( \ldots \left(
\frac{\partial \; \delta(x)}{\partial x} \cdot \delta(x) \right)
\ldots \cdot \delta(x) \right) \cdot \delta(x) \right)}$, in which 
 $\displaystyle{\frac{\partial}{\partial x}}$ and
$\delta(x)$ occur  $(p-1)$ times.

It is obvious that $[x,z]\:=\:0$, Furthermore from
\[
[y,z] \,=\, \displaystyle{\sum_{a+b=p-1,a,b \geq 0}x^{a} \delta(x) x^{b}}
\,=\, p \delta(x) x^{p-1} \,=\, 0
\]
we deduce that $z$ also commutes with $y$. Hence $z$ is in the center
of $C$.

To prove that $w$ is in the center of $C$ we use the following
 key-lemma. This lemma  will also be  used
 in the new proof of Proposition~\ref{ref-1.3-4}.

\begin{lemma} \label{ref-3.1-7}
    Let $f \in B$,
    and let $g$ be the element $\displaystyle{ \frac{\partial}{\partial
    x} \left( \frac{ \partial}{ \partial x} \left( \ldots \left(
    \frac{\partial \; f}{\partial  x} \cdot f \right) \ldots \cdot f
    \right) \cdot f \right)}$ of $B$, where both
    $\displaystyle{\frac{\partial}{\partial x}}$ and $f$ occur $(p-1)$
    times.
     Then $\displaystyle{ \frac{\partial \;g}{\partial x} \:=\:0}$.
\end{lemma}

\begin{proof} Without loss of generality we may assume that $f\neq
    0$. Define the derivation $d$ of $B$ by
    $\displaystyle{d(b) :~= \frac{ \partial b}{\partial x} \cdot f}$,
    and
    consider the differential operator $e\,=\,d^{p}\,-\,g \cdot d$ on $B$. Since
    the $p$th power of a derivation is also a derivation, it follows
    that $e$ is also a derivation.

 If we evaluate $e$ in $x$, we get
    $e(x)\,=\,d^{p}(x) \,-\,g \cdot d(x) \,=\,
    d^{p-1}(f) \,-\, g \cdot f \,=\, \displaystyle{d^{p-2} \left( \frac{\partial \;
    f}{\partial x} \cdot f \right) \,-\, g \cdot f \: = \ldots =\: f
    \cdot \frac{\partial}{\partial x} \left( \frac{ \partial}{ \partial x}
    \left( \ldots \left( \frac{\partial \; f}{\partial x} \cdot f \right)
    \ldots \cdot f \right) \cdot f \right) \,-\, g \cdot f} \,=\, 
    f \cdot g \,-\, g \cdot f \,=\, 0$ and so $e$ is identically zero
    on $B$.
  
In particular $e$ commutes with $d$. Computing with operators, we find
$0=[d,e]=[d,d^p-g\cdot d]=dg\cdot d$. Evaluating at $x$ and using the
fact that $f\neq 0$, this yields $\displaystyle{\frac{\partial \; g}{\partial
  x}=0}$.
\end{proof}

Let $y_{l}$, respectively $y_{r}$ be left, respectively right
multiplication by $y$ on $B$. Because $y_{l}$ and $y_{r}$ commute, we
see that $\displaystyle{[y,-]^{p} \,=\, \sum_{i=0}^{p} \left(
\begin{array}{c}p\\i \end{array} \right) y_{l}^{i}\;(-y_{r})^{p-i}}$ 
$=\, y_{l}^{p} \,-\, y_{r}^{p} \,=\,[y^{p},-]$.
It follows that we have $[y^{p},x] \,=\, [y,[y, \ldots ,[y ,\delta(x)]
\ldots ]]$ ($(p-1)$ times $y$) and by repeatedly using the fact that
$\displaystyle{[y,f(x)]\,=\, \frac{\partial \; f(x)}{\partial x}[y,x]
\,=\, \frac{\partial \; f(x)}{\partial x} \cdot \delta(x)}$, for all
$f(x) \in B$, we deduce, for $f(x) \,=\, \delta(x)$, $[y^{p},x] \,=\,
c_{p}(x)\,[y,x]$.

It follows that $w$ commutes with $x$. Let us prove that it also
commutes with $y$. $\displaystyle{[y,w]  \,=\, [y,c_{p}(x)]\,y
\,=\, \frac{\partial \; c_{p}(x)}{\partial x} \, [y,x] \, y}$ and
applying Lemma~\ref{ref-3.1-7} with $f \,=\, \delta(x) \in B$, we
deduce $[y,w] \,=\, 0$.
So we obtain $k[[z,w]] \subset Z(C)$.

 Let $Q(Z(C))$ and  $Q(C)$ be respectively the quotientfields of
$Z(C)$ and $C$.
Since $\{x^{a}y^{b}\;|\;0 \leq a,b \leq p-1 \}$ is a basis of $C$
over $k[[z,w]]$, we see that $C$ is free of rank $p^{2}$ over
$k[[z,w]]$. This implies that 
$p^{2}~\,=\,~\dim_{k((z,w))}Q(Z(C))~\cdot~\dim_{Q(Z(C))}Q(C)$, so 
$\dim_{Q(Z(C))}Q(C)~\in~\{1,p,p^{2}\}$.
Since $C$ is not commutative and  $\dim_{Q(Z(C))}Q(C)$ is a
square according to   \cite{Cohn}, it follows
that $\dim_{Q(Z(C))}Q(C)=p^{2}$ and furthermore that $Z(C)$ and
$k[[z,w]]$ have the same quotientfield.

As indicated above $C$ is free of rank $p^{2}$
over $k[[z,w]]$. In particular $C$ is finitely generated as a module over
$k[[z,w]]$. It follows that $Z(C)$ is also finitely generated as a
module over $k[[z,w]]$ and thus  $Z(C)$ is integral over $k[[z,w]]$.
Since 
$k[[z,w]]$ is integrally closed, it  follows that $Z(C) \:=\:
k[[z,w]]$.

\medskip

 So in order to complete the proof Conjecture~\ref{ref-1.1-1} in
 this special case, we have to
show that if $v(\delta(x))\ge 3$ then $v(c_{p}(x)) \,>\, p-1$, where $v$ is the $x$-adic valuation
on $B$. Therefore, let $c_{r}(x)$ be equal to
$\displaystyle{\frac{\partial}{\partial x}
\left( \frac{\partial}{\partial x} \left( \ldots \left(
\frac{\partial \; \delta(x)}{\partial x} \cdot \delta(x) \right)
\ldots \cdot \delta(x) \right) \cdot \delta(x) \right)}$ in which
$\displaystyle{\frac{\partial}{\partial x}}$ and $\delta(x)$ occur
$(r-1)$ times and this for all $r \geq 2$. 
\\ We prove by induction that $v(c_r(x))\ge 2(r-1)$.

 Since $v(\delta(x)) \geq 3$, $\displaystyle{v(c_{2}(x))\:=\:v \left(
\frac{\partial \; \delta(x)}{\partial x} \right) \geq 2}$, 
so we get by induction that
\\ $v(c_{r}(x))\:=\:\displaystyle{v \left( \frac{\partial}{\partial x}
\left( c_{r-1}(x) \cdot \delta(x) \right) \right)} \:=\: v(c_{r-1}(x))
\,+\, v(\delta(x)) \,-\,1 \geq 2(r-2) \,+\,3\,-\,1\,=\,2(r-1)$. So
$v(c_{p}(x)) \geq 2(p-1) > p-1$.




\section{The case where $\delta =0$ and $\sigma$ is not trivial but has finite order} \label{ref-4-8}
In this case the commutation relation between $y$ and $x$ is given by:

\begin{equation} \label{ref-4.1-9}
    y\,x\,=\,\sigma(x)\,y
\end{equation}

We will denote the order of $\sigma$ by $n$ and put 
$A\,=\,B^{\sigma}$. Let $K$, $L$ be the quotientfields of $A$, $B$ 
respectively.
We  prove that $Z(C)\,=\,k[[z,y^{n}]]$, where 
$z\,=\,x\,\sigma(x) \ldots \sigma^{n-1}(x)$.
Let us first discuss the structure of $A$.

\begin{lemma} \label{ref-4.1-10}
    $A\,=\,k[[z]]$, with $z$ as above.
\end{lemma}

\begin{proof}
    It is obvious that $A$ is a complete discrete valuation ring and 
    $k$ is a copy of its residue field. So $A$ is a formal power 
    series ring $k[[u]]$, where $u$ is a uniformizing element. Being 
    a uniformizing element, $u$ must be of the form 
    $x^{e}\;+$ higher terms, where $e$ is the ramification index.
   
Since $K$ is complete under a discrete valuation, $L$ is 
    a finite extension of $K$ and the residue class degree equals 1, 
    we conclude that $e\,=\,[L:K]\,=\,n$.
    
It is easy to see that $\sigma(x)\,=\,\zeta\,x\,+$ higher 
    terms, where $\zeta$ is an $n$th root of unity. So 
    $z\,=\,x\,\sigma(x) \ldots \sigma^{n-1}(x)$ is of the form 
    $\pm x^{n}\,+$ higher terms. Therefore $z$ is also a uniformizing 
    element and furthermore $A=k[[z]]$.
\end{proof}

It is clear that $A \subset Z(C)$ and that $y^{n}$ belongs to the 
center of $C$.
We now look at the other inclusion.

 Let $f$ be in $Z(C)$. We can write $f$, in a unique way, in 
the form $\displaystyle{\sum_{i \geq 0}a_{i}\,y^{i}}$, where $a_{i} 
\in B$.
 Since $f \in Z(C)$, we have (using (\ref{ref-4.1-9})) $0\,=\,[x,f]\,=\, 
\displaystyle{\sum_{i \geq 0} a_{i}\,(x\,-\,\sigma^{i}(x))\,y^{i}}$. 
Hence, for all $i \in \NN$, if $a_{i} \,\neq\, 0, \; 
x\,=\,\sigma^{i}(x)$, so $n$ divides $i$. On the other hand we have 
$0\,=\,[y,f]\,=\, \displaystyle{\sum_{i \geq 0}(\sigma(a_{i}) \,-\, 
a_{i}) \,y^{i+1}}$, so $\sigma(a_{i})\,=\,a_{i}$, for all $i$ in 
$\NN$, which means that $a_{i} \in A$, for all $i$ in $\NN$.
Therefore $f \in k[[z,y^{n}]]$.


  We have now proved that $Z(C)$ is a formal power series ring in the two
  variables $z,w$.  The remaining claim of
  Conjecture \ref{ref-1.1-1}  follows from the fact that if
  $\sigma(x)$ is of the form $x+\cdots$ then
\begin{itemize}
\item If $p=0$ and $\sigma$ is non-trivial then its order is infinite  (easily proved).
\item If $p>0$ and if the order  of $\sigma$ is finite then it is a power of $p$ \cite{SS}.
\end{itemize}


\section{The case where $\sigma$ and $\delta$ are non trivial and
  $\sigma$ has finite order} \label{ref-5-11}


Here we have the following commutationrelation between $y$ and $x$ :

\begin{equation} \label{ref-5.1-12}
    y \, x \,=\, \sigma(x) \, y \,+\, \delta(x)
\end{equation}

As before we denote the order of $\sigma$ by $n$ and we assume $n 
\neq 1$.
 We put $A \,=\, B^{\sigma}$ and we let $K$ and $L$ be respectevely the 
quotientfields of $A$ and  $B$.
We  extend the action of $\sigma$ and $\delta$ to $L$ and we denote
these  extended maps also by $\sigma$ and $\delta$.

 It was shown in  Lemma~\ref{ref-4.1-10}, that $A$ is 
the ring of power series over $k$ in 
\\ $z \,=\, x\,\sigma(x) \ldots 
\sigma^{n-1}(x) \,\in\, B$.

For convenience we will first work in the polynomial \"Ore extension
$S \,=\, B[y;\sigma,\delta]$.  We prove:

\begin{theorem} \label{ref-5.1-13}
    The center $Z(S)$ of $S$ is the ring of polynomials $A[w]$, 
    where $w$ is a monic (skew) polynomial of degree $n$ in $y$ with 
    coefficients in $B$. In particular, we find that $S$ is free of rank $n^{2}$ over 
    $Z(S)$.
\end{theorem}

The proof of this theorem depends on the following lemma:

\begin{lemma} \label{ref-5.2-14}
    Let $D$, $D'$ be central simple algebras of the same PI-degree 
    with centers $Z$, $Z'$, respectively. Assume that $D \subseteq D'$.
    Then $Z\subseteq Z'$ and furthermore the map $\varphi : D \otimes_{Z} Z' \rightarrow D'$, 
    defined by $\varphi(d \otimes z') \,:=\, dz'$, is an 
    isomorphism.
\end{lemma}

\begin{proof}
    Denote the PI-degree of $D$ and $D'$ by $m$.
    The PI-degree of $DZ'$ is equal to $m$ since we have 
    inclusions $D \subseteq DZ' \subseteq D'$. From $Z' \subseteq 
    Z(DZ') \subseteq DZ' \subseteq D'$ (where $Z(DZ')$ is the center 
    of $DZ'$), we deduce that $m^{2} \,=\, [DZ':Z(DZ')] \,\leq\, 
    [DZ':Z'] \,\leq\, [D':Z'] \,=\, m^{2}$ ,so $[DZ':Z'] \,=\, m^{2} 
    \,=\, [D':Z']$. This implies $DZ' \,=\, D'$ and in particular
    $Z\subseteq Z(DZ')= Z(D')=Z'$.

    We conclude that the $\varphi : D \otimes_{Z} Z' \rightarrow D'$ 
    is an epimorphism. Since $D$ is a central simple algebra, 
    the same holds for $D \otimes_{Z} Z'$. Thus $D \otimes_{Z} Z'$ is simple 
    and it follows that $\varphi$ must be an isomorphism.
\end{proof}

\begin{proof}[Proof of Theorem~\ref{ref-5.1-13}]
    Working out the identity $\delta(x \cdot f) \,=\, \delta(f \cdot 
    x)$, for all $f \in B$, we deduce:
    
    \begin{equation} \label{ref-5.2-15}
        \delta(f) \,=\, \frac{ \sigma(f) \,-\, f}{\sigma(x) \,-\, x} \cdot 
        \delta(x)
    \end{equation}
    
    This implies immediately that, if $f \in A$, then $\delta(f) \,=\, 
    0$, in other words, the polynomial ring $R \,=\, A[y]$ is a 
    commutative subring of $S$.
    \\ Now consider $S$ as right $R$-module. The rank of $S$ over $R$ 
    is $n$, since $B \,=\, k[[x]]$ is free of rank $n$ over $A \,=\, 
    k[[z]] \,=\, k[[x^{n} \,+\, \mbox{higher terms}]]$.
    \\ Left multiplication yields an injective ringhomomorphism:
    
    \begin{equation} \label{ref-5.3-16}
        S \hookrightarrow \End_{R}(S_{R})
    \end{equation}
    
    So $S$ satisfies a polynomial identity because $S$ is isomorphic 
    to a subring of the matrix ring $M_{n}(R)$, which is a PI-ring 
    since $R$ is commutative. This implies also that the PI-degree 
    of $S$ is less or equal to the PI-degree of $M_{n}(R)$ which is 
    $n$. We claim that it is exactly $n$.
     To see this, filter $S$ by $y$ degree and denote the associated 
    graded ring by $\gr S$.
     Since $\gr S \,=\, B[\overline{y};\sigma]$, we see that 
    $\gr S$ is a domain and furthermore $Z(\gr S) \,=\, 
    A[\overline{y}^{n}]$ by Section~\ref{ref-3-5}. So 
    $\gr S$ is a prime ring of rank $n^{2}$ over its center 
    which implies that its PI-degree is equal to $n$. 
    Since the PI-degree of $S \,\geq$ PI-degree of 
    $\gr S$, it now follows that the PI-degree of $S$ is exactly 
    $n$.
    
     Let $E$ be the quotientfield of $S$. As in (\ref{ref-5.3-16}) we have 
    an inclusion:
        \begin{equation} \label{ref-5.4-17}
        i : E \hookrightarrow \End_{K(y)}(E_{K(y)})
    \end{equation}
        $E$ is a central simple algebra of PI-degree $n$ and so is 
    $\End_{K(y)}(E_{K(y)})$. Hence (\ref{ref-5.4-17}) induces, by 
    Lemma~\ref{ref-5.2-14}, an isomorphism 
        \begin{equation} \label{ref-5.5-18}
        \varphi : E \otimes_{Z(E)} K(y) \hookrightarrow \End_{K(y)}(E_{K(y)})   
    \end{equation}
        defined by $\varphi(e \otimes f) \,=\, i(e) \cdot f$.
         This means that we can compute the characteristic polynomial 
    of each $e \in E$, in $\End_{K(y)}(E_{K(y)})$.
    
 Since $S$ is an \"Ore extension, it is also a maximal order by 
    \cite{MR1} and so it is closed under taking coefficients of 
    reduced characteristic polynomials.
     Using this observation we can now explicitly construct central 
    elements in the center of $S$ and the one we are interested in, is the 
    reduced norm of $y$.
    
By definition this reduced norm may be computed by taking  the image 
    of $y$ in $\End_{K(y)}(E_{K(y)})$ under (\ref{ref-5.5-18}), i.e. 
    $\varphi(y \otimes 1) \,=\, i(y)$, where $i(y)$ is left 
    multiplication by $y$, and then computing the determinant of $i(y)$ 
    in $\End_{K(y)}(E_{K(y)})$.
   
To perform this computation we need a suitable basis for $E \,/\,
    K(y)$. We pick  
    a normal basis  $\{f, \sigma(f), \ldots , \sigma^{n-1}(f) \}$ for 
    $L \,/\, K$, for some $f \in L$ in 
    \cite{Cohn}. This is still a basis for
    $E \,/\, K(y)$.



     We now compute  the matrix of $i(y)$ explicitly.
     By (\ref{ref-5.1-12}) we get, for all $0 \leq j \leq n-1$, 
    $i(y)(\sigma^{j}(f)) \,=\, \sigma^{j+1}(f) \cdot y \,+\, 
    \delta(\sigma^{j}(f))$, and since $\{f, \sigma(f), \ldots , 
    \sigma^{n-1}(f) \}$ is a basis for $L \,/\, K$, 
    $i(y)(\sigma^{j}(f)) \,=\, \sigma^{j+1}(f) \cdot y \,+\, 
    \displaystyle{\sum_{i=0}^{n-1} \sigma^{i}(f) \cdot a_{ji}}$, for 
    certain $a_{ji} \in K$.
    
This means that the matrix of $i(y) \,=\, D \,+\, Cy$, where 
    $D \,=\, (a_{ji}) \in M_{n}(K)$ and 
    \[ C = \left( \begin{array}{ccccc} 0 & 1 & 0 & \ldots & 0 \\ 0 & 
    0 & 1 & \ldots & 0 \\ \vdots & \vdots & \vdots & & \vdots \\ 0 & 
    0 & 0 & \ldots & 1 \\ 1 & 0 & 0 & \ldots & 0 \end{array} \right) \]
    the matrix of a cyclic permutation. 
Hence $\operatorname{Nrd}(y) \,=\, \det(D \,+\, Cy) \,=\, (-1)^{n+1}y^{n} \,+\, \mbox{lower terms 
    in } y$. 
    
     We now take $w \,=\, (-1)^{n+1}\Nrd(y)$. Clearly $A[w] \subset 
    Z(S)$. Since $B$ is free of rank $n$ over $A$ and $w \,=\, y^{n} 
    \,+\, \mbox{ lower terms in } y$, $S$ is free of rank $n^{2}$ 
    over $A[w]$. In particular, $Z(S)$ is integral over $A[w]$. Now 
    because $A[w] \subset Z(S) \subset S$, we know that $K(w) \subset 
    Q(Z(S)) \subset E$, where $Q(Z(S))$ is the quotientfield of $Z(S)$. 
    Since $S$ is free of rank $n^{2}$ over $A[w]$ and $E$ is a central 
    simple algebra of PI-degree $n$, the dimension of $Q(Z(S))$ over 
    $K(w)$ must be 1, so $A[w]$ and $Z(S)$ have the same quotientfield.
    
The fact that $A[w]$ is integrally closed 
    and that $Z(S)$ is integral over $A[w]$ now implies that $A[w] \,=\, 
    Z(S)$.
\end{proof}



In the next proposition we will obtain more information on the 
element $w$ constructed in the above theorem.
 Let $v$ be the $x$-adic valuation on $B$.
\begin{proposition} \label{ref-5.3-19}
    Assume that $v(\delta(x)) \,=\, a$.
    \\ If $w \,=\, y^{n} \,+\, \displaystyle{\sum_{i=0}^{n-1} 
    f_{i}(x)\,y^{i}}$, then for $i > 0$ we have $v(f_{i}) \,\geq\, 
    (a-1)(n-i)$. Furthermore there exists an element $q_0(z)\in
    k[[z]]$ such that $v(f_0+q_0(z))\ge (a-1)n$.
\end{proposition}

In the proof of this proposition we need the result of the following 
lemma:

\begin{lemma} \label{ref-5.4-20}
    If $f \in B$, then $\displaystyle{v \left( \frac{\sigma(f) \,-\, 
    f}{\sigma(x) \,-\, x} \right) \geq v(f) \,-\, 1}$.
\end{lemma}

\begin{proof}
    Put $r=v(f)$.
\begin{case} $r \geq 1$
    \\ Put $h \,=\, \sigma(x) \,-\, x$, then we get 
    $\displaystyle{\frac{\sigma(f) \,-\, f}{\sigma(x) \,-\, x} \,=\, 
    \frac{f(\sigma(x)) \,-\, f(x)}{\sigma(x) \,-\, x} \,=\, 
    \frac{f(x+h) \,-\, f(x)}{h}}$. Since $f(x) \,=\, 
    \displaystyle{\sum_{i=r}^{+\infty} a_{i}x^{i}}$, for certain 
    $a_{i} \in k$ with $a_{r} \neq 0$, it is easy to see that 
    \[\frac{f(x+h) \,-\, f(x)}{h} \,=\, \sum_{i=0}^{+\infty} \left( 
    \sum_{j=r}^{+\infty} a_{j} \psi_{i,j} h^{j-i-1} \right) x^{i} \]
    where 
$\psi_{ij}=\begin{cases} 0&\text{if $i\ge j$}\\
\frac{j!}{i!(j-i)!} &\text{if $i<j$}
\end{cases} 
$.
    
So $\displaystyle{v \left( \frac{\sigma(f) \,-\, f}{\sigma(x) 
    \,-\, x} \right) \,=\, v \left( \frac{f(x+h) \,-\, 
    f(x)}{h}\right)}\ge \min_i((r-i-1)v(h)+i)\ge r-1$
    since $v(h) \ge v(x)\ge 1$.
\end{case}
\begin{case} $r=0$.
    \\ In this case we get that $f(x) \,=\, 
    \displaystyle{\sum_{i=0}^{+\infty} a_{i}x^{i}}$, for certain 
    $a_{i} \in k$ with $a_{0} \neq 0$.
     Since $\sigma$ is an automorphism which is also $k$-linear, it 
    follows that $\displaystyle{v \left( \frac{\sigma(f) \,-\, 
    f}{\sigma(x) \,-\, x} \right) \,=\, v \left( \frac{\sigma(g) \,-\, 
    g}{\sigma(x) \,-\, x} \right)}$, where $g \,=\, 
    \displaystyle{\sum_{i=1}^{+\infty} a_{i}x^{i}}$. Since $v(g) \geq 
    1$, we get by applying Case 1, $\displaystyle{v \left( \frac{\sigma(f) \,-\, 
    f}{\sigma(x) \,-\, x} \right)} \geq v(g) \,-\, 1 \geq 0 \geq v(f) 
    \,-\, 1$. \qed
\end{case}
\def\qed{}\end{proof}
We return now to the proof of Proposition~\ref{ref-5.3-19}.
\begin{proof}[Proof of Proposition~\ref{ref-5.3-19}]
    Put $\overline{y} \,=\, x^{-a+1}y$. If we multiply (\ref{ref-5.1-12})
    on the left with $x^{-a+1}$, we 
    obtain
    \begin{equation} \label{ref-5.6-21}
        \overline{y}\,x \,=\, \sigma(x) \, \overline{y} \,+\, x^{-a+1} \, 
        \delta(x)
    \end{equation}
    Consider the ring $\overline{S} \,=\, 
    B[\overline{y};\sigma,\overline{\delta}]$, where 
    $\overline{\delta}$ is the $\sigma$-derivation of $B$ defined by 
    $\overline{\delta}(b) \,=\, x^{-a+1} \, \delta(b)$.
     We clearly have inclusions $S \subset \overline{S} \subset 
    L[y;\sigma,\delta]$.
    
 Applying Theorem~\ref{ref-5.1-13} to $\overline{S}$, we find 
    that $\overline{S}$ has a central element $\overline{w}$ of the 
    form 
    \begin{equation} \label{ref-5.7-22}
        \overline{w} \,=\, \overline{y}^{n} \,+\, \sum_{i=0}^{n-1} g_{i}(x) 
        \overline{y}^{i}
    \end{equation}
    with $g_{i}(x) \in B$.
     Verifying the commutationrelation of $x^{-a+1}$ and $y$, we 
    find 
    \begin{equation} \label{ref-5.8-23}
        y\,x^{-a+1} \,=\, \sigma(x^{-a+1}) \, y \,+\, \delta(x^{-a+1})
    \end{equation}
    For all $f \in B$, we get by (\ref{ref-5.2-15}) and Lemma~\ref{ref-5.4-20} that $v(\delta(f)) \,=\, \displaystyle{v \left( \frac{ 
    \sigma(f) \,-\, f}{\sigma(x) \,-\, x} \cdot \delta(x) \right)}$ 
    \\ $\,=\, \displaystyle{v \left( \frac{\sigma(f) \,-\, f}{\sigma(x) \,-\, x} 
    \right) \,+\, v(\delta(x))} \geq v(f) \,-\, 1 \,+\, a$. 
    \\ In particular, it follows that $\delta(x^{-a+1}) \in B$.
    
 Using (\ref{ref-5.8-23}), we can rewrite $\overline{w}$ in the 
    following form
    \[\overline{w} = z^{-a+1}y^{n} + h_{0}(x) + \sum_{i=1}^{n-1} (x 
    \cdot \sigma(x) \cdot \ldots \cdot \sigma^{i-1}(x))^{-a+1} 
    h_{i}(x) y^{i} \]
    where, for all $0 \leq i \leq n-1$, we have $h_{i}(x) \in B$ and 
    with $z$ the element of $A$ defined in Section~\ref{ref-4-8}.
    
 Multiplying $\overline{w}$ with $z^{a-1}$, we get the element 
    \[y^{n} \,+\, z^{a-1}h_{0}(x) \,+\, \sum_{i=1}^{n-1} 
    (\sigma^{i}(x) \cdot \ldots \cdot 
    \sigma^{n-1}(x))^{a-1}h_{i}(x)y^{i}\] 
    which we will denote by $w'$.
   
 Let us write $p_{0}(x)$ for $z^{a-1}h_{0}(x)$ and $p_{i}(x)$ for  
 $(\sigma^{i}(x) \cdot \ldots 
    \cdot \sigma^{n-1}(x))^{a-1}h_{i}(x)$, for all $1 
    \leq i \leq n-1$.
     Since $v(p_{0}(x)) \,=\, (a-1) \, v(z) \,+\, v(h_{0}(x)) \geq 
    (a-1) n \geq 0$ and, for all $1 \leq i \leq n-1$, $v(p_{i}(x)) 
    \,=\, (a-1) \displaystyle{\left( \sum_{j=i}^{n-1} 
    v(\sigma^{j}(x)) \right)} \,+\, v(h_{i}(x)) \geq (a-1)(n-i) \geq 
    0$, we see that $w'$ belongs to $S$. $w'$ is also a central 
    element of $\overline{S}$, so it follows that $w'$ is a central 
    element of $S$.
     This means that $w'$ has to be of the form
    \begin{equation} \label{ref-5.9-24}
        w' \,=\, \sum q_{i}(z) w^{i}
    \end{equation}
    since by Theorem~\ref{ref-5.1-13}, we know that $Z(S) \,=\, 
    A[w] \,=\, k[[z]][w]$.
    
By looking at the degree of $y$, we can reduce (\ref{ref-5.9-24}) to 
    $w' \,=\, q_{0}(z) \,+\, q_{1}(z) \, w$ and if we look at the 
    coefficient of $y^{n}$, we see that $q_{1}(z) \,=\, 1$.
    Hence $f_{i}(x) \,=\, p_{i}(x)$, for all $1 \leq i \leq n-1$, 
    which implies that $v(f_{i}(x)) \geq (a-1)(n-i)$. Furthermore
    $p_0(x)=q_0(z)+f_0(x)$ which implies $v( q_0(z)+f_0(x))\ge (a-1)n$. 
\end{proof} 

\begin{corollary} \label{ref-5.5-25}
    Let $C$ be the formal power series ring 
    $k[[x]][[y;\sigma,\delta]]$, where  \mbox{$v(\delta(x)) \geq 3$}. Let $n$ 
    be the order of $\sigma$. Then the center of $C$ is equal to 
    $k[[z,w]]$, where $z \,=\, x^{n} \,+\, \varphi(x)$ and $w \,=\, 
    y^{n} \,+\, \theta(x,y)$ with $\varphi,\theta$ containing only 
    terms in $x$, $y$ of total degree $>n$.
\end{corollary}

\begin{proof}
Let $M\subset S$ be the twosided ideal generated by $x,y$. Clearly $C$
is equal to the $M$-adic completion of $S$. Let $m$ be the maximal
ideal of $Z(S)$ generated by $z,w$. It is easy to see that 
\begin{gather*}
M^{2N}\subset mS \subset M\\
m^aS\cap Z(S)=m^a
\end{gather*}
Thus the completion of  $Z(S)$ at the induced topology coincides with
the completion at the $m$-adic topology, which is $k[[z,w]]$.  Since
$S\subset C$ the PI-degree of $C$ is $\ge n$. On the other hand, using
the properties of completion every identity in $S$ vanishes in $C$. So
the PI-degree of $C$ is exactly $n$. Since $Z(C)\supset k[[z,w]]$,
$\rk_{Z(C)} C=n^2$  and  $k[[z,w]]$ is integrally closed, we prove
exactly as before that $Z(C)= k[[z,w]]$. 
\end{proof}
To complete the proof of  Theorem~\ref{ref-1.2-3} we use the fact that
in characteristic $p>0$
 the order of $\sigma$ is a power of $p$
\cite{SS1}.


\section{A new proof of Proposition~\ref{ref-1.3-4}} \label{ref-6-26}
Let $k$ be a field of characteristic $p > 1$
and consider the field $k(t_{1}, \dots ,t_{p-1})$, where $t_{1}, 
\ldots , t_{p-1}$ are variables. Let  $f\,=\, \displaystyle{ 
\sum_{i=1}^{p-1} f_{i}\,t_{i}} \in k(t_{1}, \ldots ,t_{p-1})[x]$ be arbitrary.

Since $k(t_{1}, \ldots ,t_{p-1})$ is also a field of 
characteristic $p$
it follows from Lemma~\ref{ref-3.1-7} that $f$ satisfies
\begin{equation}
\label{ref-6.1-27}
 \frac{\partial^2}{\partial x^2} \left( \frac{\partial}
{\partial x} \left( \ldots \left( \frac{\partial \; f}{\partial x} 
\cdot f \right) \ldots \cdot f \right) \cdot f \right) =0
\end{equation}
where $\displaystyle{\frac{\partial}{\partial x}}$ occurs $p$ times and 
$f$ occurs 
$(p-1)$ times.

It is clear that $\displaystyle{ \frac{\partial \;f}{\partial x} 
\,=\, \sum_{i=1}^{p-1} \frac{\partial \; f_{i}}{\partial x} \cdot t_{i}}$. 
Taking the coefficient of $t_{1} 
\cdot \ldots \cdot t_{p-1}$ in \eqref{ref-6.1-27} we get 
\[\sum_{\sigma \in S_{p-1}} \frac{\partial^{2}}{\partial x^{2}} 
\left( \frac{\partial}{\partial x} \left( \ldots \left( 
\frac{\partial \; f_{\sigma(1)}}{\partial x} \cdot f_{\sigma(2)} 
\right) \ldots \cdot f_{\sigma(p-2)} \right) \cdot f_{\sigma(p-1} 
\right) \,=\,0 \]
for all polynomials $f_{i}$ over a field $k$ of characteristic $p >0$.



Consider the following expression in the variables $f_{1}, \ldots 
,f_{p-1}$:

\begin{equation} \label{ref-6.2-28}
    \sum_{\sigma \in S_{p-1}} \left[ \frac{\partial^{2}}{\partial x^{2}} 
    \left( \frac{\partial}{\partial x} \left( \ldots \left( 
    \frac{\partial \; f_{\sigma(1)}}{\partial x} \cdot f_{\sigma(2)} 
    \right) \ldots \cdot f_{\sigma(p-2)} \right) \cdot 
    f_{\sigma(p-1)} \right) \right.
\end{equation}
\[ \left. -\; \frac{\partial^{p} \; f_{\sigma(1)}}{\partial x^{p}} 
    \cdot f_{\sigma(2)} \cdot \ldots \cdot f_{\sigma(p-1)} \right] \]


(\ref{ref-6.2-28}) has the following properties:

\begin{enumerate}
    
    \item[(a)] (\ref{ref-6.2-28})$\:=\:0$, if $f_{1}, \ldots ,f_{p-1}$ are 
    polynomials over a field $k$ of characteristic $p >0$.
    
    \item[(b)] Over any field, we may rewrite (\ref{ref-6.2-28}) in the form
    
    \begin{equation} \label{ref-6.3-29}
        \sum_{0\le u_{1}, \ldots ,u_{p-1}\le p-1} a_{u_{1} \ldots u_{p-1}} \: 
        \frac{\partial^{u_{1}} \; f_{1}}{\partial x^{u_{1}}} \cdot \ldots 
        \cdot \frac{\partial^{u_{p-1}} \; f_{p-1}}{\partial x^{u_{p-1}}}
    \end{equation}
    such that $a_{u_{1} 
    \ldots u_{p-1}} \in \ZZ$.
\end{enumerate}    
    
Using these properties we will prove that the coefficients of 
(\ref{ref-6.3-29}) are multiples of $p$.

Define for  $q,\: n \in \NN$ the symbolic $n$th 
power $q^{(n)}$ of $q$ as follows:
\[q^{(n)} \:=\: \left\{ \begin{array}{lll} 1 & & \mbox{if } n\,=\,0 \\
q \, (q\,-\,1) \ldots (q\,-\,n\,+\,1) & & \mbox{if } n \geq 1 
\end{array} \right. \]
Now let $(q_i)_{i=1,\ldots,p-1}\in \NN$ be arbitrary and put $f_{i} \,=\,
x^{q_{i}}$.
Then it is easy to see that 
(\ref{ref-6.3-29}) equals 
\[\sum_{u_{1}, \ldots ,u_{p-1}} a_{u_{1} \ldots u_{p-1}} \: 
q_{1}^{(u_{1})} \ldots q_{p-1}^{(u_{p-1})} \: x^{q_{1}-u_{1}} \ldots 
x^{q_{p-1}-u_{p-1}}\]

Since (\ref{ref-6.3-29}) is zero in $k$ by  property (a) we 
deduce:
\begin{equation} \label{ref-6.4-30}
    \sum_{u_{1}, \ldots ,u_{p-1}} a_{u_{1} \ldots u_{p-1}} \: 
    q_{1}^{(u_{1})} \ldots q_{p-1}^{(u_{p-1})} \;=\; 0
\end{equation}
in $k$.

Let $X$ be the $k$-vectorspace of all functions 
$h \,:\, k^{p-1} \rightarrow k$. By \cite{BS}
\[
\left\{ x_{1}^{u_{1}} \ldots x_{p-1}^{u_{p-1}} \; | \; \mbox{for all } 
1 \leq i \leq p-1, \; u_{i} \leq p-1 \right\}
\]
 is a basis for $X$. 
We may transform these `normal' monomials into 
`symbolic' monomials by a triangular matrix whose determinant is equal 
to 1. It follows that
\[\left\{ x_{1}^{(u_{1})} \ldots x_{p-1}^{(u_{p-1})} \; | \; 
\mbox{for all } 1 \leq i \leq p-1, \; u_{i} \leq p-1 \right\}
\]
is also 
a basis for $X$.

Since (\ref{ref-6.4-30}) holds for all $q_{1}, \ldots ,q_{p-1} \in \NN$, 
this implies that 
\[\sum_{u_{1}, \ldots ,u_{p-1}} a_{u_{1} \ldots u_{p-1}} \: 
x_{1}^{(u_{1})} \dots x_{p-1}^{(u_{p-1})} \;=\; 0 \]
in $k$.
We conclude that the coefficients $a_{u_{1} \ldots u_{p-1}}$ are 
zero in $k$ and hence they are divisible by $p$, as elements of $\ZZ$.
 
 Let us look now at the difference of (\ref{ref-6.2-28}) and (\ref{ref-6.3-29}), i.e.
\begin{equation} \label{ref-6.5-31}
    \sum_{\sigma \in S_{p-1}} \left[ 
    \frac{\partial^{2}}{\partial x^{2}} \left( 
    \frac{\partial}{\partial x} \left( \ldots \left( 
    \frac{\partial \; f_{\sigma(1)}}{\partial x} \cdot f_{\sigma(2)} 
    \right) \ldots \cdot f_{\sigma(p-2)} \right) \cdot 
    f_{\sigma(p-1)} \right) \right. 
\end{equation}
\[ \left. -\; \frac{\partial^{p} \; f_{\sigma(1)}}{\partial x^{p}} 
    \cdot f_{\sigma(2)} \cdot \ldots \cdot f_{\sigma(p-1)} \right] 
    \;-\; \sum_{u_{1}, \ldots ,u_{p-1}} 
    a_{u_{1} \ldots u_{p-1}} \: 
    \frac{\partial^{u_{1}} \; f_{1}}{\partial x^{u_{1}}} 
    \cdot \ldots \cdot 
    \frac{\partial^{u_{p-1}} \; f_{p-1}}{\partial x^{u_{p-1}}} \]
By definition (\ref{ref-6.5-31}) is equal to zero over any field with a 
derivation.  We will consider \eqref{ref-6.5-31} over the complex numbers $\CC$.
Let $(v_i)_{i=1,\ldots,p-1}\in \CC$ and put $f_{i}\,=\,e^{v_{i}x}$.
We deduce that 
\[ \sum_{\sigma \in S_{p-1}} \left[ v_{\sigma(1)} \, 
(v_{\sigma(1)}\,+\,v_{\sigma(2)}) \ldots (v_{\sigma(1)}\,+ \ldots +\, 
v_{\sigma(p-1)})^{2} \; 
e^{(v_{\sigma(1)} \,+ \ldots +\, v_{\sigma(p-1)})\,x} \right. \]
\[ \left. -\; v_{\sigma(1)}^{p} \; 
e^{(v_{\sigma(1)} \,+ \ldots +\, v_{\sigma(p-1)})\,x} \right] \;-\; 
\sum_{u_{1}, \ldots ,u_{p-1}} a_{u_{1} \ldots u_{p-1}} \: 
v_{1}^{u_{1}} \ldots v_{p-1}^{u_{p-1}} \: e^{(v_{1} \,+ \ldots +\, 
v_{p-1})\,x} \;=\;0 \]

If we divide this by $e^{(v_{1} \,+ \ldots +\,v_{p-1})\,x}$, we get, 
for all $v_{1}, \ldots ,v_{p-1} \in \CC$
\[ \sum_{\sigma \in S_{p-1}} (\, v_{\sigma(1)} \, 
(v_{\sigma(1)} \,+\, v_{\sigma(2)}) \ldots 
(v_{\sigma(1)} \,+ \ldots +\, v_{\sigma(p-1)})^{2} 
\;-\; v_{\sigma(1)}^{p} \,) \]
\[ -\; \sum_{u_{1}, \ldots ,u_{p-1}} a_{u_{1} \ldots u_{p-1}} \: 
v_{1}^{u_{1}} \ldots v_{p-1}^{u_{p-1}} \;=\; 0 \]
So the polynomial
\[ \sum_{\sigma \in S_{p-1}} (\, x_{\sigma(1)} \, 
(x_{\sigma(1)} \,+\, x_{\sigma(2)}) \ldots 
(x_{\sigma(1)} \,+ \ldots +\, x_{\sigma(p-1)})^{2} 
\;-\; x_{\sigma(1)}^{p} \,) \]
\[ -\; \sum_{u_{1}, \ldots ,u_{p-1}} a_{u_{1} \ldots u_{p-1}} \: 
x_{1}^{u_{1}} \ldots x_{p-1}^{u_{p-1}} \]
is identically zero.

If we reduce this modulo $p$, we deduce that 
\[ \left[ \sum_{\sigma \in S_{p-1}} x_{\sigma(1)} \, 
(x_{\sigma(1)} \,+\, x_{\sigma(2)}) \ldots 
(x_{\sigma(1)} \,+ \ldots +\, x_{\sigma(p-1)}) \right] \; 
(x_{1} \,+ \ldots +\, x_{p-1}) \]
\[ \equiv\; x_{1}^{p} \,+ \ldots +\, x_{p-1}^{p} \;\equiv\;
(x_{1} \,+ \ldots +\, x_{p-1})^{p} \hspace{1cm} (\mbox{mod}\;p) \]
Hence Proposition~\ref{ref-1.3-4} is proved.



%\bibliography{mybibs}
%\bibliographystyle{amsabbrv}

\ifx\undefined\bysame
\newcommand{\bysame}{\leavevmode\hbox to3em{\hrulefill}\,}
\fi
\begin{thebibliography}{1}

\bibitem{BS1}
G.~Baron and A.~Schinzel, {\em An extension of {W}ilson's theorem}, C. R. Math.
  Rep. Acad. Sci. Canada {\bf 1} (1978/79), no.~2, 115--118.

\bibitem{BS}
Z.~I. Borevitch and I.~R. Shafarevitch, {\em Number theory}, Academic Press,
  1966.

\bibitem{Cohn}
P.~M. Cohn, {\em Algebra}, John Wiley {\&} Sons, 1982.

\bibitem{MR1}
G.~Maury and J.~Raynaud, {\em Ordres maximaux au sens de {K}. {A}sano},
  Springer, Berlin, 1980.

\bibitem{SS1}
S.~Sen, {\em On automorphisms of local fields}, Ann. of Math. (2) {\bf 90}
  (1969), 33--46.

\bibitem{SS}
S.~P. Smith and J.~T. Stafford, {\em Regularity of the 4-dimensional {S}klyanin
  algebra}, Compositio Math. {\bf 83} (1992), 259--289.

\bibitem{VdBVG}
M.~Van~den Bergh and M.~Van~Gastel, {\em Graded modules of {G}elfand-{K}irillov
  dimension one over three-dimensional {A}rtin-{S}chelter regular algebras}, J.
  Algebra {\bf 196} (1997), 251--282.

\end{thebibliography}


\begin{comment}
\ifx\undefined\bysame
\newcommand{\bysame}{\leavevmode\hbox to3em{\hrulefill}\,}
\fi
\begin{thebibliography}{10}

\bibitem{1}
M.~Van~den Bergh and M.~Van~Gastel, {\em Graded modules of {G}elfand-{K}irillov
  dimension one over three-dimensional {A}rtin-{S}chelter regular
algebras}, \ldots

\bibitem{2}
G.~Baron and A.~Schinzel,\ldots

\bibitem{3}
Cohn, vol3

\bibitem{4}
Shankar Sen

\bibitem{5}
Maury en Raynaud prop 2.3

\bibitem{6}
Borevitch en Safarevitch

\end{thebibliography}
\end{comment}
\end{document}



%%% Local Variables:
%%% mode: latex
%%% TeX-master: t
%%% End: 



